\documentclass[11pt,a4paper]{article}
% Deutsch
\usepackage[german]{babel} % deutsch und deutsche Rechtschreibung
\usepackage[utf8]{inputenc} % Unicode Text 
\usepackage[T1]{fontenc} % Umlaute und deutsches Trennen
\usepackage{textcomp} % Euro
\usepackage[hyphens]{url}
% statt immer Ab\-schluss\-ar\-beit zu schreiben
% einfach hier sammeln mit -. 
\hyphenation{Ab-schluss-ar-beit}
% Vorsicht bei Umlauten und Bindestrichen
\hyphenation{Ver-st\"ar-ker-aus-gang}
 % eigene Hyphenations, die für das Dokument gelten
\usepackage{amssymb} % Symbole
\usepackage{emptypage} % Wirklich leer bei leeren Seiten

%% Fonts, je ein kompletter Satz an Optionen

% Times New Roman, gewohnter Font, ok tt und serifenlos
%\usepackage{mathptmx} 
%\usepackage[scaled=.95]{helvet}
%\usepackage{courier}

% Palatino mit guten Fonts für tt und serifenlos
\usepackage{mathpazo} % Palatino, mal was anderes
\usepackage[scaled=.95]{helvet}
\usepackage{courier}

% New Century Schoolbook sieht auch nett aus (macht auch tt und serifenlos)
%\usepackage{newcent}

% Oder default serifenlos mit Helvetica 
% ich kann es nicht mehr sehen ...
%\renewcommand{\familydefault}{\sfdefault}

% ein bisschen eine bessere Verteilung der Buchstaben...
\usepackage{microtype}

% Bilder und Listings
\usepackage{graphicx} % wir wollen Bilder einfügen
\usepackage{subfig} % Teilbilder
\usepackage{wrapfig} % vielleicht doch besser vermeiden
\usepackage{listings} % schöne Quellcode-Listings
% ein paar Einstellungen für akzeptable Listings
\lstset{basicstyle=\ttfamily, columns=[l]flexible, mathescape=true, showstringspaces=false, numbers=left, numberstyle=\tiny}
\lstset{language=python} % und nur schöne Programmiersprachen ;-)
% und eine eigene Umgebung für Listings
\usepackage{float}
\newfloat{listing}{htbp}{scl}[chapter]
\floatname{listing}{Listing}

% Seitenlayout
\usepackage[paper=a4paper,width=14.8cm,left=30mm,right=30mm,height=23cm]{geometry}
\usepackage{setspace}
\linespread{1.15}
\setlength{\parskip}{0.5em}
\setlength{\parindent}{0em} % im Deutschen Einrückung nicht üblich, leider

% Seitenmarkierungen 
\newcommand{\phv}{\fontfamily{phv}\fontseries{m}\fontsize{9}{11}\selectfont}
\usepackage{fancyhdr} % Schickere Header und Footer
\pagestyle{fancy}
\renewcommand{\chaptermark}[1]{\markboth{#1}{}}
%\fancyhead[L]{\phv \leftmark}
\fancyhead[RE,LO]{\phv \nouppercase{\leftmark}}
\fancyhead[LE,RO]{\phv \thepage}
% Unten besser auf alles Verzichten
%\fancyfoot[L]{\textsf{\small \kurztitel}}
\fancyfoot[C]{\ } % keine Seitenzahl unten
%\fancyfoot[R]{\textsf{\small Technische Informatik}}

% Theorem-Umgebungen
\newtheorem{definition}{Definition}[chapter]
\newtheorem{satz}{Satz}[chapter]
\newtheorem{lemma}[satz]{Lemma} % gleicher Zähler wie Satz
\newtheorem{theorem}{Theorem}[chapter]
\newenvironment{beweis}[1][Beweis]{\begin{trivlist}
\item[\hskip \labelsep {\textit{#1 }}]}{\end{trivlist}}
\newcommand{\qed}{\hfill \ensuremath{\square}}

%% Quellen
% Eine Alternative wäre Quellen in Literatur und Online-Quellen
% zu teilen
% \usepackage{bibtopic} 

% Hochschule Logo, noch nicht perfekt
\usepackage{hsmalogo}

% Spezialpakete
\usepackage{epigraph}
\setlength{\epigraphrule}{0pt} % kein Trennstrich

% damit wir nicht so viel tippen müssen, nur für Demo 
\usepackage{blindtext} 

% ifthen für sperrvermerk
\newif\ifsperrvermerk

\usepackage{blindtext}
\usepackage{xfrac}

\title{Hochschule Mannheim}
\author{Nicolas Pawelka}
\date{14.10.2022}


\begin{document}
\maketitle

\begin{abstract}

\end{abstract}
\tableofcontents

\section{Einleitung}
Hallo das ist meine Einleitung.

Hier kommt viel Information hin.

Letzter Absatz in der Einleitung. In \ref{sec:Leitbild} ist das Leitbild der Hochschule Mannheim mit dem Schwerpunkt Lehre und...
\section{Leitbild}\label{sec:Leitbild}
\Blindtext[1]

\subsection{Lehre und Forschung}
\Blindtext[2]
\subsection{Internationalisierung}
\Blindtext[2]
\subsection{Wissenschaftliche Weiterbildung}
\Blindtext[2]
\subsection{Regionaler und überregionaler Wissens- und Technologietransfer}
\Blindtext[2]

\section{Zahlen und Fakten}

\begin{equation}
 \sum_{i=1}^n i^2 = \frac{1}{6} \cdot n \cdot (n+1) \cdot (2n+1)
\end{equation}
\begin{equation}\label{eq:1}
\lim_{n \to \infty} \sum_{k=0}^n q^k = \lim_{n \to \infty} \frac{1 - q^{n+1}}{1-q}
\end{equation}
\begin{equation}\label{eq:2}
 = \frac{1}{1-q}
\end{equation}

Dabei gilt (\ref{eq:1}) für q < 1 wegen der geometrischen SUmmenformel. Die Grenzwertbetrachtung liefert dann (\ref{eq:2}).

Wie wir in Tabelle \ref{tab:numbers} sehen ist die Hochschule Mannheim mit nur 5.200 Studierenden eine kleine Hochschule.
\begin{table}[H]
\begin{center}
\begin{tabular}{|l|l|r|}
\hline
\textbf{Fakultäten} & \textbf{Studiengänge} & \textbf{Studierende} \\\hline\hline
9 & 39 (insgesamt) & 5200
\\\hline
\end{tabular}
\end{center}
\caption{Zahlen der HS Mannheim}\label{tab:numbers}
\end{table}
In der Tabelle \ref{tab:Neue} können wir  sehen das viel mehr Leute im WS anfangen im Gegensatz zum Sommersemester...

\begin{table}[H]
\begin{center}
\begin{tabular}{|l||l|r|}
\hline
\textbf{Neuimmatrikulierte} & \textbf{Wintersemester} & \textbf{Sommersemester}\\\hline\hline
1894 & 1250 & 644
\\\hline
\end{tabular}
\end{center}
\caption{Anzahl der Neuimmatrikulierten}\label{tab:Neue}
\end{table}
 
 Wir haben ganz viele tolle Leute die Dinge erklären wie Tabelle \ref{tab:Mitarbeiter} zeigt..
\begin{table}[H]
\begin{center}
\begin{tabular}{|l||l|r|}
\hline
\textbf{Professoren} & \textbf{Lehrbeauftragte} & \textbf{Sprachlehrkräfte}\\\hline\hline
173 & 162 & 5
\\\hline
\end{tabular}
\end{center}
\caption{Anzahl der Mitarbeiter}\label{tab:Mitarbeiter}
\end{table}
\section{Impressionen}
\section{Fazit}

Zitieren geht ganz einfach \cite{Beningo} oder so \cite{Lehner}

\bibliographystyle{plain}
\bibliography{hsma}

\end{document}