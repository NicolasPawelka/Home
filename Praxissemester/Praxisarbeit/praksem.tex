\documentclass[11pt,a4paper]{report} 
% dasselbe Template wie Thesis mit nur leichten Anpassungen
% Nehmen Sie das Thesis-Template für die Thesis!
% Lesen Sie Hinweise zum Umgang mit LaTeX und zum Schreiben
% von Berichten im Thesis-Template nach
% => Moodle => PraxissemesterThesis => LaTeXThesis.zip
%    https://moodle.hs-mannheim.de/course/view.php?id=2500


% Für doppelseitigen Ausdruck (nur bei > 60 Seiten sinnvoll)
% \usepackage{ifthen}
% \setboolean{@twoside}{true}
% \setboolean{@openright}{true} 

\include{preamble} % alle Pakete und Einstellungen

% Hier anpassen 
\newcommand{\autor}{Nicolas Pawelka}
\newcommand{\matrikelnummer}{2120666}
\newcommand{\fachsemester}{5} % im wie vielten Semester waren Sie?
\newcommand{\studiengang}{Medizintechnik}
%\newcommand{\studiengang}{Technische Informatik}
%\newcommand{\studiengang}{Informationstechnik}
\newcommand{\firma}{NewTec GmbH}
\newcommand{\standort}{Mannheim}
\newcommand{\abteilung}{Team embedded Software}
\newcommand{\betreuer}{Julia Kammerer}
\newcommand{\pbeginn}{4.9.2023}
\newcommand{\pende}{29.02.2024}
\newcommand{\tage}{123} % arbeitstagerechner verwenden!
\newcommand{\titel}{Bericht zum praktischen Studiensemester}
\newcommand{\kurztitel}{Praxisbericht}
\sperrvermerktrue % Kommentar am Anfang der Zeile löschen für Sperrvermerk

% Wenn jemand unbedingt ein Glossar will, die nächsten drei Zeilen...
%\usepackage{glossaries} % oder schlimmer mit [toc], damit es im TOC auftaucht
%\makeglossaries
%\input{glossar} % In dieser Datei die Einträge definieren
% und noch ganz unten printglossaries auskommentieren
% Damit jetzt ein Glossar gezeigt wird noch \gls{label} verwenden

\begin{document}
\include{vorspann} % Titelseite, Erklärungen, etc.

\begin{abstract}
  Beim führenden Anbieter von Fallen aller Art ist die
  Produktentwicklung geprägt von Kreativität und höchsten
  Qualitätsansprüchen, die schon in der Forschungs- und
  Entwicklungsabteilung umfassende Tests und Messungen
  nach sich zieht.
  
  Bei ausgiebigen Materialprüfungen konnte bei Federn für
  Steingewichte $>5$ Tonnen eine ungewünschte Zielabweichung
  bestätigt werden. Beim Aufbau der Prüfumgebung, insbesondere der
  Nachverfolgung der Flugbahn mit Kameras wurde die Genauigkeit
  der Messung mit kleineren Steinen zunächst bestimmt und dann
  die Abweichung in Abhängigkeit der Masse belegt.
  Eine deutliche Verbesserung von 5 Sekunden auf 0,1 Sekunden
  konnte bei der Auslösung der Raketentriebwerke erreicht werden.
  Durch die massive Zunahme der Anzahl der Sensormessungen war
  das eingesetzte Sortierverfahren nicht mehr in der Lage die
  Berechnungen rechtzeitig abzuschließen.
  Durch den Einsatz eines effizienten, stabilen Standardsortierverfahrens
  ist dieser Teil des Gesamtaufwands bis zu 10000 Sensorwerten von bis
  zu 5 Sekunden auf immer unter 10 ms auf dem eingesetzten Mikrocontroller
  ACME (ACME Controlled Micro Engine) 503 geschrumpft.
  Auch bei den bisher mechanisch ausgeführten Langwaffen (Bogen) konnte
  durch den Einsatz von ausgeklügelter Elektronik die Anwendbarkeit
  deutlich verbessert werden.
  Das entwickelte System meldet gegebenenfalls einen 416-Fehler, falls
  das Ziel für einen erfolgreichen Einsatz zu weit entfernt ist.
  Ein versuchter Abschuss bei eingerichteter Sperre, der vorher zu
  erheblichen Schäden am Gerät geführt hat, ergibt jetzt einen
  benutzerfreundlichen 423-Fehler.
  Falls das eingesetzte Geschoss (Chip-kodiert) nicht für die
  Abschussgeschwindigkeit freigegeben ist oder die Zielführung kein
  relevantes Ziel verifizieren kann, wird dies mit einem
  428-Fehler angezeigt.
\end{abstract}

\tableofcontents

\chapter{Einführung} \label{chap:einf}



\chapter{Produkte und Prüfverfahren} \label{chap:ppv}

Trotz Fokus auf den Anwendungsbereich Fallen wird ein weiter Bereich
von technischer Infrastruktur im Produktangebot abgedeckt.
Die eingesetzten Prüfverfahren testen meist eine mechanische
und zeitliche Genauigkeit.

\section{Schleudern, Raketen und Bögen} \label{sec:was}

\blindtext[2]
\blindtext[2]

\section{Mechanische Genauigkeit} \label{sec:mec}

\blindtext[2]
\blindtext[1]
\blindtext[2]

\section{Zeitliche Abweichungen} \label{sec:time}

\blindtext[1]
\blindtext[2]
\blindtext[1]

\chapter{Genauigkeit von Schleudern} \label{chap:sling}

Die Genauigkeit von Schleudern hängt von den beiden Faktoren
Eins und Zwei ab und wird mit Drei oder Vier gemessen.
Eins kombiniert X und Y, um qualitative Aussagen über den
strukturellen Aufbau zu machen.
Im Gegensatz dazu setzt sich Zwei aus A, B und C zusammen
und erlaubt eine quantitative Festlegung.
Das Verfahren Drei liefert mit wenig Aufwand einen ersten
Anhaltspunkt, ob weitere detaillierte Messungen notwendig sind.
Mit dem aufwendigen Verfahren Vier können wir dann Toleranzen
auf $\pm 0.1\%$ Abweichung bestimmen.

\section{Eins}

\blindtext[3]
\blindtext[1]
\blindtext[2]
\blindtext[3]

\section{Zwei}

\blindtext[1]
\blindtext[2]
\blindtext[4]
\blindtext[2]

\section{Drei}

\blindtext[3]
\blindtext[2]
\blindtext[2]
\blindtext[1]

\chapter{Steuern von Raketentriebwerken} \label{chap:rocket}

Raketentriebwerke werden hauptsächlich mechanisch gesteuert.
Nur das Anstoßen von Zustandsübergängen Zwei wird elektronisch
getriggert. Drei ist eine übliche Realisierung.

\section{Eins}

\blindtext[2]
\blindtext[4]
\blindtext[3]
\blindtext[3]
\blindtext[2]

\section{Zwei}

\blindtext[3]
\blindtext[1]
\blindtext[3]
\blindtext[4]
\blindtext[1]

\section{Drei}

\blindtext[2]
\blindtext[3]
\blindtext[2]
\blindtext[4]

\chapter{Benutzbarkeit von Bögen} \label{chap:bow}

Text\ldots

\section{Eins}

\blindtext[1]
\blindtext[2]
\blindtext[3]
\blindtext[2]
\blindtext[1]

\section{Zwei}

\blindtext[2]
\blindtext[2]
\blindtext[3]
\blindtext[2]

\section{Drei}

\blindtext[3]
\blindtext[5]
\blindtext[3]
\blindtext[2]


\chapter{Fazit} \label{chap:fazit}

\blindtext[2]
\blindtext[1]


\newpage

% Listen wenn überhaupt ans Ende und nicht an den Anfang.
% Meist ist das aber unnötig.
%\listoffigures % Liste der Abbildungen 
%\begingroup % aahh nicht noch ein pagebreak
%\let\clearpage\relax %
%\listoftables % Liste der Tabellen
%\endgroup

% Glossar kommt auch ans Ende
%\glsaddall % das fügt alle Glossar-Einträge ein
%\printglossaries % nicht vergessen "makeglossaries praksem" aufzurufen
%\newpage

\addcontentsline{toc}{chapter}{Literaturverzeichnis}
\bibliographystyle{plain} % Literaturverzeichnis
\bibliography{praksem}
% \bibliography{praksem,online} # wenn man zwei Dateien hätte

% Das wäre die Alternative mit geteilten Quellen (preamble muss auch
% angepasst werden) und die Literatur muss in die Datei praksem.bib
% und die Online-Quellen müssen in die Datei online.bib.
%\begin{btSect}{praksem} % mit bibtopic Quellen trennen
%\section*{Literaturverzeichnis}
%\addcontentsline{toc}{chapter}{Literaturverzeichnis}
%\btPrintCited
%\end{btSect}
%\begin{btSect}{online}
%\section*{Online-Quellen}
%\addcontentsline{toc}{chapter}{Online-Quellen}
%\btPrintCited
%\end{btSect}
% dann ab und zu "bibtex praksem1" und "bibtex praksem2" aufrufen

\end{document}
;;; Local Variables:
;;; ispell-local-dictionary: "de_DE-neu"
;;; End:
