\documentclass[11pt,a4paper]{article}
\input{preamble}
\usepackage{blindtext}
\usepackage{xfrac}

\title{Hochschule Mannheim}
\author{Nicolas Pawelka}
\date{14.10.2022}


\begin{document}
\maketitle

\begin{abstract}

\end{abstract}
\tableofcontents

\section{Einleitung}
Hallo das ist meine Einleitung.

Hier kommt viel Information hin.

Letzter Absatz in der Einleitung. In \ref{sec:Leitbild} ist das Leitbild der Hochschule Mannheim mit dem Schwerpunkt Lehre und...
\section{Leitbild}\label{sec:Leitbild}
\Blindtext[1]

\subsection{Lehre und Forschung}
\Blindtext[2]
\subsection{Internationalisierung}
\Blindtext[2]
\subsection{Wissenschaftliche Weiterbildung}
\Blindtext[2]
\subsection{Regionaler und überregionaler Wissens- und Technologietransfer}
\Blindtext[2]

\section{Zahlen und Fakten}

\begin{equation}
 \sum_{i=1}^n i^2 = \frac{1}{6} \cdot n \cdot (n+1) \cdot (2n+1)
\end{equation}
\begin{equation}\label{eq:1}
\lim_{n \to \infty} \sum_{k=0}^n q^k = \lim_{n \to \infty} \frac{1 - q^{n+1}}{1-q}
\end{equation}
\begin{equation}\label{eq:2}
 = \frac{1}{1-q}
\end{equation}

Dabei gilt (\ref{eq:1}) für q < 1 wegen der geometrischen SUmmenformel. Die Grenzwertbetrachtung liefert dann (\ref{eq:2}).

Wie wir in Tabelle \ref{tab:numbers} sehen ist die Hochschule Mannheim mit nur 5.200 Studierenden eine kleine Hochschule.
\begin{table}[H]
\begin{center}
\begin{tabular}{|l|l|r|}
\hline
\textbf{Fakultäten} & \textbf{Studiengänge} & \textbf{Studierende} \\\hline\hline
9 & 39 (insgesamt) & 5200
\\\hline
\end{tabular}
\end{center}
\caption{Zahlen der HS Mannheim}\label{tab:numbers}
\end{table}
In der Tabelle \ref{tab:Neue} können wir  sehen das viel mehr Leute im WS anfangen im Gegensatz zum Sommersemester...

\begin{table}[H]
\begin{center}
\begin{tabular}{|l||l|r|}
\hline
\textbf{Neuimmatrikulierte} & \textbf{Wintersemester} & \textbf{Sommersemester}\\\hline\hline
1894 & 1250 & 644
\\\hline
\end{tabular}
\end{center}
\caption{Anzahl der Neuimmatrikulierten}\label{tab:Neue}
\end{table}
 
 Wir haben ganz viele tolle Leute die Dinge erklären wie Tabelle \ref{tab:Mitarbeiter} zeigt..
\begin{table}[H]
\begin{center}
\begin{tabular}{|l||l|r|}
\hline
\textbf{Professoren} & \textbf{Lehrbeauftragte} & \textbf{Sprachlehrkräfte}\\\hline\hline
173 & 162 & 5
\\\hline
\end{tabular}
\end{center}
\caption{Anzahl der Mitarbeiter}\label{tab:Mitarbeiter}
\end{table}
\section{Impressionen}
\section{Fazit}

Zitieren geht ganz einfach \cite{Beningo} oder so \cite{Lehner}

\bibliographystyle{plain}
\bibliography{hsma}

\end{document}