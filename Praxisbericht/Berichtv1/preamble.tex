% Deutsch
\usepackage[german]{babel} % deutsch und deutsche Rechtschreibung
\usepackage[utf8]{inputenc} % Unicode Text 
\usepackage[T1]{fontenc} % Umlaute und deutsches Trennen
\usepackage{textcomp} % Euro
\usepackage[hyphens]{url}
% statt immer Ab\-schluss\-ar\-beit zu schreiben
% einfach hier sammeln mit -. 
\hyphenation{Ab-schluss-ar-beit}
% Vorsicht bei Umlauten und Bindestrichen
\hyphenation{Ver-st\"ar-ker-aus-gang}
 % eigene Hyphenations, die für das Dokument gelten
\usepackage{amssymb} % Symbole
\usepackage{emptypage} % Wirklich leer bei leeren Seiten

%% Fonts, je ein kompletter Satz an Optionen

% Times New Roman, gewohnter Font, ok tt und serifenlos
%\usepackage{mathptmx} 
%\usepackage[scaled=.95]{helvet}
%\usepackage{courier}

% Palatino mit guten Fonts für tt und serifenlos
\usepackage{mathpazo} % Palatino, mal was anderes
\usepackage[scaled=.95]{helvet}
\usepackage{courier}

% New Century Schoolbook sieht auch nett aus (macht auch tt und serifenlos)
%\usepackage{newcent}

% Oder default serifenlos mit Helvetica 
% ich kann es nicht mehr sehen ...
%\renewcommand{\familydefault}{\sfdefault}

% ein bisschen eine bessere Verteilung der Buchstaben...
\usepackage{microtype}

% Bilder und Listings
\usepackage{graphicx} % wir wollen Bilder einfügen
\usepackage{subfig} % Teilbilder
\usepackage{wrapfig} % vielleicht doch besser vermeiden
\usepackage{listings} % schöne Quellcode-Listings
% ein paar Einstellungen für akzeptable Listings
\lstset{basicstyle=\ttfamily, columns=[l]flexible, mathescape=true, showstringspaces=false, numbers=left, numberstyle=\tiny}
\lstset{language=python} % und nur schöne Programmiersprachen ;-)
% und eine eigene Umgebung für Listings
\usepackage{float}
\newfloat{listing}{htbp}{scl}[chapter]
\floatname{listing}{Listing}

% Seitenlayout
\usepackage[paper=a4paper,width=14.8cm,left=30mm,right=30mm,height=23cm]{geometry}
\usepackage{setspace}
\linespread{1.15}
\setlength{\parskip}{0.5em}
\setlength{\parindent}{0em} % im Deutschen Einrückung nicht üblich, leider

% Seitenmarkierungen 
\newcommand{\phv}{\fontfamily{phv}\fontseries{m}\fontsize{9}{11}\selectfont}
\usepackage{fancyhdr} % Schickere Header und Footer
\pagestyle{fancy}
\renewcommand{\chaptermark}[1]{\markboth{#1}{}}
%\fancyhead[L]{\phv \leftmark}
\fancyhead[RE,LO]{\phv \nouppercase{\leftmark}}
\fancyhead[LE,RO]{\phv \thepage}
% Unten besser auf alles Verzichten
%\fancyfoot[L]{\textsf{\small \kurztitel}}
\fancyfoot[C]{\ } % keine Seitenzahl unten
%\fancyfoot[R]{\textsf{\small Technische Informatik}}

% Theorem-Umgebungen
\newtheorem{definition}{Definition}[chapter]
\newtheorem{satz}{Satz}[chapter]
\newtheorem{lemma}[satz]{Lemma} % gleicher Zähler wie Satz
\newtheorem{theorem}{Theorem}[chapter]
\newenvironment{beweis}[1][Beweis]{\begin{trivlist}
\item[\hskip \labelsep {\textit{#1 }}]}{\end{trivlist}}
\newcommand{\qed}{\hfill \ensuremath{\square}}

%% Quellen
% Eine Alternative wäre Quellen in Literatur und Online-Quellen
% zu teilen
% \usepackage{bibtopic} 

% Hochschule Logo, noch nicht perfekt
\usepackage{hsmalogo}

% Spezialpakete
\usepackage{epigraph}
\setlength{\epigraphrule}{0pt} % kein Trennstrich

% damit wir nicht so viel tippen müssen, nur für Demo 
\usepackage{blindtext} 

\usepackage{natbib}

% ifthen für sperrvermerk
\newif\ifsperrvermerk

%Personalisierte Packages
\usepackage{minted}
\usepackage{enumitem}
\usepackage{subcaption}
\usepackage{lipsum}
\usepackage[inkscapelatex=false]{svg}
